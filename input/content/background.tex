\chapter{Background}
\label{chap:background}

This chapter lays the foundation on which rest of the thesis
is built. It describes the CUDA programming model which the
code generation is based upon along with how it maps to a
GPU -- a concrete instance of a parallel architecture.
Finally, it describes the programming language Futhark and
its optimising compiler.


\section{The CUDA Programming Model}
\label{sec:cuda}

When people talk about CUDA they often refer to it as a
language, e.g., ``I am writing CUDA code'', which is
neither completely wrong nor correct. To use NVIDIAs own
words: \emph{CUDA is a general purpose parallel computing
  platform and programming model}.

The platform part is a software layer including language
extensions, compiler, debugger, drivers, and runtime, that
enables programmers to write programs reaping the power of
GPGPUs. In this thesis we will use the CUDA language
extension to C/C++, but other languages and directive-based
approaches are supported, e.g., Fortran and OpenACC.

The programming model part is often overlooked because the
language extensions implementing the programming model put a
lot of focus on hardware. To get a high-level understanding
of the CUDA programming model it is sufficient to know that
a GPU is a physical device with a separate memory space from
the host and that it can execute multithreaded code.

The CUDA programming model is a heterogeneous model which
refers to the fact that more than one kind of processor is
used in order to ... [WIKI REFERENCE]. \todo{Write some
  more} (Introduce \emph{host} and \emph{device}.)  Usually,
the structure of a simple host program is as follows:
%
% If you use this you should probably cite it
% https://devblogs.nvidia.com/easy-introduction-cuda-fortran/
\begin{enumerate}
  \item Declare and allocate host and device memory.
  \item Initialize host memory.
  \item Transfer data from host to device.
  \item Upload and execute program on device.
  \item Transfer data from device to host.
\end{enumerate}
%
This is also illustrated in Figure~\ref{fig:heterogeneous}.
%
%
\begin{figure}
\begin{center}
\begin{tikzpicture}[
      every node/.style={minimum size=1cm-\pgflinewidth, outer sep=0pt, font=\scriptsize}]
% Top - Host
\filldraw[fill=light-gray, draw=black] (0,2) rectangle (6,3);
\filldraw[fill=dark-gray, draw=black] (0,2) rectangle (0.5,3);
\filldraw[fill=dark-gray, draw=black] (1.5,2) rectangle (3,3);
\filldraw[fill=dark-gray, draw=black] (5,2) rectangle (6,3);
\node[fit={(0,2)(6,3)}] (upper) {};
\node[anchor=east, at=(upper.west)] (hostlab) {Host};
% Bottom - Device
\filldraw[fill=light-gray, draw=black] (0,0) rectangle (6,1);
\filldraw[fill=dark-gray, draw=black] (0.5,0) rectangle (1.5,1);
\filldraw[fill=dark-gray, draw=black] (3,0) rectangle (5,1);
\node[fit={(0,0)(6,1)}] (lower) {};
\node [anchor=east, at=(lower.west)] (devlab) {Device};
% Arrows
\draw[-latex] (0.5,2) -- node[above, rotate=90, align=center] {Data \\ + Code} (0.5,1);
\draw[-latex] (1.5,1) -- node[above, rotate=90] {Data} (1.5,2);
%
\draw[-latex] (3,2) -- node[above, rotate=90, align=center] {Data \\ + Code} (3,1);
\draw[-latex] (5,1) -- node[above, rotate=90] {Data} (5,2);
% Time line
\draw[-latex] ([yshift=0.25cm]upper.north west) -- node[above] {Time} ([yshift=0.25cm]upper.north east);
\draw[latex-latex] ([xshift=0.25cm]upper.north east) -- node[below, rotate=90] {Data} ([xshift=0.25cm]lower.south east);
\end{tikzpicture}
\end{center}
\caption{Heterogeneous programming. The host is transferring
  code and data to the device, and the device transfers back
  data to the host. Here the host is launching two
  kernels. The areas colored with a darker color is
  indicating where the computation is. The host is always in
  control of the device.}
\label{fig:heterogeneous}
\end{figure}
%

%

Even though we argued that the programming model is often
overlooked due to the language extensions hardware focus we
will use the C/C++ language extension as a vehicle for
understanding the following parts of the programming model. \todo{Only the parts relevant to this thesis}
% These parts can be grouped into three main categories
% describing threads, memory, and how threads may cooperate.

\subsection{Kernels}

Programs uploaded to a device are also called kernels, not
to be confused with operating system kernels or kernels from
image processing.

When telling the device to execute a kernel the user must
specify how many threads to use, arguments to the kernel,
and possibly the amount of memory needed. Each thread then
sequentially executes the instructions contained in the
kernel code, but threads are executed in parallel.

Kernels are launched by using the special kernel-execution
syntax indicated by triple chevrons:
%
\begin{verbatim}
  kernel_name<<< ... >>>( ... )
\end{verbatim}
%
where \texttt{kernel\_name} is a C function identifier, and
what goes into between the parentheses are normal function
arguments. The arguments that goes into the chevrons specify
how to group the threads, which we will describe now.


\subsection{Thread Hierarchy}

%
%
\begin{figure}
\begin{center}
\begin{tikzpicture}[]
  \newcommand\rectsep{1pt}
  \newcommand\mysep{0.1}
  \newcommand\myxshift{4}
  \newcommand\rectbox[5]{%{x0}{y0}{x1}{y1}{Label}{label p0}{label p1}
    %\draw[fill=dark-gray] (#1cm+\rectsep,#2cm+\rectsep) rectangle node{#5} +(#3cm-2*\rectsep,#4cm-2*\rectsep);
    \draw[fill=dark-gray] (#1+\mysep+\myxshift, #2+\mysep) node (bob) {} rectangle node (spelunking) {\footnotesize #5} (#3-\mysep+\myxshift, #4-\mysep) node (notbob) {};
  }
%% Grid
\node[draw, fill=light-gray, fit={(-10.0+\myxshift, 0.0)(-5.50+\myxshift, 4.5)}] (upper) {};
%% Bottom row
\rectbox{-10.0}{0.0}{-8.5}{1.5}{(0,0)};
\rectbox{-8.50}{0.0}{-7.0}{1.5}{(1,0)};
\rectbox{-7.00}{0.0}{-5.5}{1.5}{(2,0)};
%% Middle row
\rectbox{-10.0}{1.5}{-8.5}{3.0}{(0,1)};
\rectbox{-8.50}{1.5}{-7.0}{3.0}{(1,1)};
\rectbox{-7.00}{1.5}{-5.5}{3.0}{(2,1)};
%% Top row
\rectbox{-10.0}{3.0}{-8.50}{4.5}{(0,2)};
\rectbox{-8.50}{3.0}{-7.00}{4.5}{(1,2)};
\rectbox{-7.00}{3.0}{-5.50}{4.5}{(2,2)};
%% Shadow nodes
\node[anchor=south west] (ld) at (-7.00+\mysep+\myxshift, 1.5+\mysep, 0) {};
\node[anchor=north west] (lu) at (-7.00+\mysep+\myxshift, 3.0-\mysep, 0) {};
\node[anchor=south east] (rd) at (-5.50-\mysep+\myxshift, 1.5+\mysep, 0) {};
\node[anchor=north east] (ru) at (-5.50-\mysep+\myxshift, 3.0-\mysep, 0) {};

%% Arrows from grid to thread block
\newcommand\myyshift{0.15}
\node[draw, fit={(0,0+\myyshift,0)(3,3+\myyshift,-3)}] (cubenode) {}; % x,y,z
\draw[dashed] (ld.south west) -- (cubenode.south west);
\draw[dashed] (lu.north west) -- (cubenode.north west);
\draw[dashed] (rd.south east) -- (cubenode.south east);
\draw[dashed] (ru.north east) -- (cubenode.north east);
\node[draw, fit={(0,0+\myyshift,0)(3,3+\myyshift,-3)}, fill=dark-gray] {}; % x,y,z
%\node[anchor=south] at ([yshift=0.15cm]cubenode.north) {Thread block};

% Fill cuboid
\filldraw[light-gray] (0,0+\myyshift,0) rectangle (3,3+\myyshift,0); % front
\filldraw[light-gray] (3,0+\myyshift,-3) -- (3,3+\myyshift,-3) -- (3,3+\myyshift,0) -- (3,0+\myyshift,0); % right
\filldraw[light-gray] (0,3+\myyshift,-3) -- (0,3+\myyshift,0) -- (3,3+\myyshift,0) -- (3,3+\myyshift,-3); % top
% Cuboid
\foreach \x in {0,...,3}
{   \draw (0,  \x+\myyshift, 0) -- (3,\x+\myyshift, 0);   % horizontal lines - front
    \draw (\x, 0+\myyshift,  0) -- (\x ,3+\myyshift, 0); % vertical lines - front
    \draw (3,\x+\myyshift, -3) -- (3,\x+\myyshift, 0); % depth - right
    \draw (\x ,3+\myyshift, -3) -- (\x ,3+\myyshift, 0); % depth - top
    \draw (3,0+\myyshift, -\x ) -- (3,3+\myyshift, -\x); % vertical lines - right
    \draw (0,3+\myyshift, -\x ) -- (3,3+\myyshift, -\x); % horizontal lines - top
};
% Arrows
\draw[-latex]   (0.5, 0.5+\myyshift, 0) to [out=0, in=160] (5.0, 0.5+\myyshift, 0);
\node[right] at (5.0, 0.5+\myyshift, 0) {{\footnotesize Thread (0, 0, 0)}};
%
\draw[-latex]   (3.0, 1.5+\myyshift, -1.5) to [out=30, in=160] (5.0, 1.5+\myyshift, 0);
\node[right] at (5.0, 1.5+\myyshift, 0) {{\footnotesize Thread (2, 1, 1)}};
%
\draw[-latex]   (1.5, 3.0+\myyshift, -1.5) to [out=20, in=140] (5.0, 2.5+\myyshift, 0);
\node[right] at (5.0, 2.5+\myyshift, 0) {{\footnotesize Thread (1, 2, 1)}};
%
\end{tikzpicture}
\end{center}
\caption{Thread hierarchy. Here we have displayed a grid
  (left) and emphasized thread block (2,1) (right), along
  with individual threads in that block. In total, this
  constitutes a kernel.}
\label{fig:thread_hierarchy}
\end{figure}
%

%
When specifying the number of threads the user must also
specify how to group these threads. We will call this
grouping or structuring for the thread space. The thread
space is split into three categories, namely threads,
blocks, and grids, where each category is enclosed in its
successor.

At the lowest level we have threads which are organised in
thread blocks or just blocks. Thread blocks have, as their
name suggests, three dimensions, x, y, and z, and all
dimensions must be greater than or equal to one. Next, a
collection of blocks are organised in a grid.  In a similar
fashion, grids are two-dimensional with x and y dimensions,
which all also must be greater than or equal to one.
Finally, we have a kernel which, as mentioned above, must
specify the thread space, i.e., grid and block size. Thus,
we have the mnemonic: \emph{a kernel is executed in a grid
  of blocks of threads}.

The kernel-execution syntax takes two arguments between the
chevrons, the first specifying the grid dimensions and the
second specifying the block dimensions.

Each thread launched then has a three-dimensional vector
corresponding to its conceptual place in the block and the
blocks place in the grid. The thread and block vectors
vector can be accessed through the built-in functions
\texttt{threadIdx} and \texttt{blockIdx}, and the vector
elements can be accessed by specifying the dimension. For
example, \texttt{threadIdx.x} gives the threads position in
the blocks x dimension, and \texttt{blockIdx.x} gives the
blocks position in the grids x dimension. Similiarly, each
thread can access the sizes of the dimensions for both
blocks and grids, e.g., \texttt{blockDim.x} and
\texttt{gridDim.x} for the length of the blocks and grids x
dimension, respectively.

Currently, the size of thread blocks are restricted to 1024
threads no matter its dimensions, but multiple equal-sized
blocks can be launched. Furthermore, blocks are required to
execute independently, hence the code is scalable with
hardware. We will get into this in
Section~\ref{sec:architecture}, but first we will describe
how threads may save and access data.


\subsection{Memory Hierarchy}

\todo{Maybe some graphics?}
% https://tex.stackexchange.com/questions/110522/how-to-elegantly-create-a-pyramid-hierarchy-in-tikz

% Here, under Local Memory:
% https://docs.nvidia.com/cuda/cuda-c-programming-guide/index.html#device-memory-accesses

% Additionally, each thread also have local memory which is
% private to that thread and is as slow as global memory.
% This memory space is used for register spilling. It is
% cached like global memory.

% Remember, we are still talking about a programming model
% so no specific implementation should be mentioned.

% Recall that blocks are required to execute independently

% It is mentioned in the intro to this section now, but you
% may repeat it here: host and device have separate memory
% spaces

Like for CPUs we can illustrate the memory hierarchy of a
GPU as a pyramid with the slowest and largest type of memory
at the bottom, and the fastest and scarcest type at the top.
Roughly, we have three main categories ranging from slowest
to fastest, namely global memory, shared memory, and
registers.

The large and slow global memory must be allocated and
initialised by specialised CUDA functions, e.g.,
\texttt{cudaMalloc()} and \texttt{cudaMemcpy()} which
allocates memory on the device and copies memory between the
host and the device, respectively. The duration of global
memory is inter-kernel meaning that it is persistent across
kernels launched by the same application, and thus it can be
used to communicate between threads in different
grids. Global memory is free'd by calling
\texttt{cudaFree()}.

Shared memory is much faster than global memory and is
allocated on a per-block basis, i.e., threads from the same
block access this same shared memory when allocated. The
lifetime of shared memory is the same of the block, thus it
can only be used to communicate between threads from the
same block. Shared memory is allocated either statically by
specifiying the amount in the kernel or dynamically using
the kernel-execution syntax.

The fastest type of memory is registers and they are private
to each thread. Registers are very scarce and is used for
storing variables, but one should be careful as the compiler
may spill them to thread-local memory which are as slow as
global memory.

% https://docs.nvidia.com/cuda/cuda-c-programming-guide/index.html#memory-fence-functions
% https://en.wikipedia.org/wiki/Memory_ordering
Finally, the CUDA programming model assumes a weakly-ordered
memory model. This means that the order in which data
written to global or shared memory, and others, may not be
the same order in which other threads observe the data being
written. Such reordering is generated by the compiler or by
the CPU at runtime and may be used for optimising bandwith
utilisation.  In a single threaded program this does not
matter -- in fact it happens all the time. For multithreaded
programs, like GPU kernels, this can cause erroneous
programs. The CUDA programming model provides memory
functions to force an ordering on memory accesses. For
example, \texttt{\_\_threadfence()} will do the work, but it
depends on not reading cached values but actual memory. This
is ensured by the volatile keyword. \todo{Not done. And Should volatile be explained here?}

% https://docs.nvidia.com/cuda/cuda-c-programming-guide/index.html#volatile-qualifier
% Volatile

\subsection{Synchronization Functions}

We will use only two synchronisation functions that work on
different levels.

The most coarse grained, \texttt{cudaThreadSynchronize()},
is called from the host and forces the host to wait for the
device to finish. This function can be used if the results
of one kernel, say \texttt{kernelA}, are required by another
subsequent kernel, \texttt{kernelB}:
%
\begin{verbatim}
kernelA<<<...>>>(...);
cudaThreadSynchronize();
kernelB<<<...>>>(...);
\end{verbatim}
%
On a per-block level we have \texttt{\_\_syncthreads()}. Divergent code.


\subsection{Atomic Functions}

As the title says, describe atomic operators:
\texttt{atomicAdd}, \texttt{atomicCAS}, and
\texttt{atomicExch}. Does not work as memory fences.
%
\begin{itemize}
  \item Ikke alle arkitekturer har atomiske operationer, men
    vi regner med at når NVIDIA implementerer det, så følger
    resten trop.
\end{itemize}
%

\section{GPGPU Architecture}
\label{sec:architecture}

% SIMT - This is implementation specific

Herein I think we should describe the architecture and how
the CUDA programming model maps to hardware.


\section{Obtaining Good Performance on GPUs}
\label{sec:performance_on_gpus}

In this section one could describe various ways to optimise
CUDA code, e.g., coalescing, and other factors that have an
impact on performance. (Kind of summarising / combining what
we have just learned in the previous two sections.)
%
\begin{itemize}
  \item Avoiding thread divergence.
  \item I guess that coalescing minimises warp divergence
    (because we increment in blockdim-chunks, and only one
    warp can be cut in ``half'').
\end{itemize}
%

Imagine a bus width of 512 bit (64 bytes) and a warp size of
32. Then if the accesses are to contigious memory, then
you'll have $\frac{32*4}{64} = 2$ memory transactions.


\section{Futhark}
\label{sec:futhark}

In this section we should describe the properties of Futhark
that we base the design of our language construct upon.
%% NOTES
\begin{itemize}
  \item Futhark introduces a programming model that builds
    on the vocabulary of functional programming, by using
    second order array combinators (SOACS) to express bulk
    operations on arrays.

  \item Futhark can be compiled to efficient GPU code -- but
    it (compiler or language?) is hardware agnostic in
    nature (?).

  \item Futhark is both a programming langauge and an
    optimising compiler. If it is not clear from the context
    we will explicitly mention which one we are referring
    to.

  \item We focus on the GPU pipeline.

  \item
\end{itemize}
%


\subsection{Overview of Compiler Phases}

%
\begin{figure}
\begin{center}
%%%% Compiler overview
\begin{tikzpicture}[
    every node/.style={font=\tiny},
    main node/.style={rectangle, draw, fill=light-gray, font=\tiny,
      minimum width=width("Internalisation"),
      minimum height=2\baselineskip},
    inner node/.style={rectangle, anchor=north east, font=\tiny},
    arrow node/.style={}
]
%% Main nodes
% Front-end
  \node[main node] (1) {Lexer};
  \node[inner node, at=(1.north east)] {1};

  \node[main node] (2) [below=of 1, align=center] {Parser};
  \node[inner node, at=(2.north east)] {2};

  \node[main node] (3) [below=of 2] {Typechecker};
  \node[inner node, at=(3.north east)] {3};
% Middle-end
  \node[main node] (4) [below=of 3] {Internalisation};
  \node[inner node, at=(4.north east)] {4};

%  \node[main node] (5) [left=of 4] {Typechecker};

  \node[main node] (6) [left=2cm of 4, align=center] {Transformations \\ + Optimisations};
  \node[inner node, at=(6.north east)] {5};

%  \node[main node] (7) [left=of 6] {Typechecker};

  \node[main node] (8) [left=2cm of 6] {IL code gen.};
  \node[inner node, at=(8.north east)] {6};
% Back-end
  \node[main node] (9) [above=of 8, align=center] {IL code gen. \\ (Kernels)};
  \node[inner node, at=(9.north east)] {7};

  \node[main node] (10) [above=of 9, align=center] {IL code gen. \\ (OpenCL calls)};
  \node[inner node, at=(10.north east)] {8};

  \node[main node] (11) [above=of 10, align=center] {OpenCL \\ code gen.};
  \node[inner node, at=(11.north east)] {9};

%% Shadow nodes
  \node [above=of 1, align=center] (start) {Source \\ code};
  \node [above=of 11, align=center] (end) {OpenCL \\ program};

%% Arrows
% Front-end
\draw [->] (start) -- (1);
\draw [->] (1) -- node[left] {Tokens} (2);
\draw [->] (2) -- node[left, align=center, name=src] {AST: \\ \phantom{NoInfo}} node[right, align=left] {Source: \\ NoInfo} (3);
% Middle-end
\draw [->] (3) -- node[left, align=center] {AST: \\ \phantom{Info}} node[right, align=left] {Source: \\Info} (4);
\draw [->] (4) -- node[above, align=center, name=soacs] {AST:} node[below] {Core: SOACS} (6);
\draw [->] (6) -- node[above, align=center, name=expmem] {AST:} node[below] {Core: ExpMem} (8);
% Back-end
\draw [->] (8) -- node[right, align=center] {IL: \\ (dunno)} (9);
\draw [->] (9) -- node[right, align=center] {IL: \\ Kernels} (10);
\draw [->] (10) -- node[right, align=center, name=opencl] {IL: \\ OpenCL} (11);
\draw [->] (11) -- (end);

%% Enclosing boxes
% Front-end
\node[draw, dashed, rounded corners, color=black, fit={(1)(3)(src)}
] (frontend) {};
\node [anchor=south, at=(frontend.west), rotate=90] (frontlab) {Front end};
% Middle-end
\node[draw, dashed, rounded corners, color=black, fit={(4)(8)(soacs)(expmem)}
] (middle) {};
\node [anchor=south, at=(middle.north)] (midlab) {Middle};
% Back-end
\node [draw, dashed, rounded corners, color=black, fit={(9)(11)(opencl)},
] (backend) {};
\node [anchor=north, at=(backend.east), rotate=90] (midlab) {Back end};
\end{tikzpicture}
%

\end{center}
\caption{Overview of compiler stages.}
\label{fig:compiler_overview}
\end{figure}
%
%
\begin{figure}
\begin{center}
%%%% Compiler overview
\begin{tikzpicture}[
    every node/.style={font=\tiny},
    main node/.style={rectangle, draw, fill=light-gray, font=\tiny,
      minimum width=width("Defunctionaliser"),
      minimum height=2\baselineskip},
    inner node/.style={rectangle, anchor=north east, font=\tiny},
    node distance=5mm and 5mm
]
%% Main nodes
% Front-end
  \node[main node] (1) {Defunctoriser};
  \node[inner node, at=(1.north east)] {4a};

  \node[main node] (2) [right=of 1, align=center] {Monomorphiser};
  \node[inner node, at=(2.north east)] {4b};

  \node[main node] (3) [below=of 2] {Defunctionaliser};
  \node[inner node, at=(3.north east)] {4c};

  \node[main node] (4) [left=of 3] {Internaliser};
  \node[inner node, at=(4.north east)] {4d};

%% Shadow nodes
  \node [left=of 1, align=center] (start) {AST: \\ Source: Info};
  \node [left=of 4, align=center] (end) {AST: \\ Core: SOACS};

%% Arrows
% Front-end
\draw [->] (start) -- (1);
\draw [->] (1) -- (2);
\draw [->] (2) -- (3);
\draw [->] (3) -- (4);
\draw [->] (4) -- (end);

%% Enclosing boxes
% Front-end
\node[draw, dashed, rounded corners, color=black, fit={(1)(2)(3)(4)}
] (frontend) {};
\node [anchor=south, at=(frontend.north)] (frontlab) {Internaliser};
\end{tikzpicture}
%

\end{center}
\caption{Overview of internaliser.}
\label{fig:internaliser}
\end{figure}
%
%
\begin{figure}
\begin{center}
%%%% Compiler overview
\begin{tikzpicture}[
    every node/.style={font=\tiny},
    main node/.style={rectangle, draw, fill=light-gray, font=\tiny,
      minimum width=width("Explicit Allocations"),
      minimum height=2\baselineskip},
    inner node/.style={rectangle, anchor=north east, font=\tiny},
    node distance=5mm and 8mm
]
%% Main nodes
% SOACS
  \node[main node] (51a) {Simplify};
  \node[inner node, at=(51a.north east)] {5.1a};

  \node[main node] (51b) [below=of 51a, align=center] {Fuse SOACS};
  \node[inner node, at=(51b.north east)] {5.1b};

  \node[main node] (51c) [below=of 51b] {(Other)};
  \node[inner node, at=(51c.north east)] {5.1c};
% Kernels
  \node[main node] (52a) [right=of 51a] {Extract Kernels};
  \node[inner node, at=(52a.north east)] {5.2a};

  \node[main node] (52b) [below=of 52a, align=center] {Babysit Kernels};
  \node[inner node, at=(52b.north east)] {5.2b};

  \node[main node] (52c) [below=of 52b] {Tile loops};
  \node[inner node, at=(52c.north east)] {5.2c};

  \node[main node] (52d) [below=of 52c] {Simplify};
  \node[inner node, at=(52d.north east)] {5.2d};
% ExplicitMemory
  \node[main node] (53a) [right=of 52a] {Explicit Allocations};
  \node[inner node, at=(53a.north east)] {5.3a};

  \node[main node] (53b) [below=of 53a, align=center] {Simplify};
  \node[inner node, at=(53b.north east)] {5.3b};

  \node[main node] (53c) [below=of 53b] {Double buffer};
  \node[inner node, at=(53c.north east)] {5.3c};

  \node[main node] (53d) [below=of 53c] {(Other)};
  \node[inner node, at=(53d.north east)] {5.3d};
%% Shadow nodes
  \node [above=1.25cm of 51a, align=center] (start) {AST: \\ Core: SOACS};
%  \node [right=of 53a, align=center] (end) {AST: \\ Core: ExplicitMemory};
  \node [below=0.60cm of 53d, align=center] (end) {AST: \\ Core: ExplicitMemory};
%% Arrows
% SOACS
\draw [->] (start) -- (51a);
\draw [->] (51a) -- (51b);
\draw [->] (51b) -- (51c);
% Kernels
\draw [->] (51c.east) -| + (0.4, 0) |- (52a.west);
\draw [->] (52a) -- (52b);
\draw [->] (52b) -- (52c);
\draw [->] (52c) -- (52d);
% ExplicitMemory
\draw [->] (52d.east) -| + (0.4, 0) |- (53a.west);
\draw [->] (53a) -- (53b);
\draw [->] (53b) -- (53c);
\draw [->] (53c) -- (53d);
%\draw [->] (53d.east) -| + (0.4, 0) |- (end.west);
\draw [->] (53d) -- (end);

%% Enclosing boxes
% SOACS
\node[draw, dashed, rounded corners, color=black, fit={(51a)(51b)(51c)}
] (soacs) {};
\node [anchor=south east, at=(soacs.north)] (soacslab) {SOACS};
% Kernels
\node[draw, dashed, rounded corners, color=black, fit={(52a)(52b)(52c)(52d)}
] (kernels) {};
\node [anchor=south, at=(kernels.north)] (kernelslab) {Kernels};
% ExplicitMemory
\node[draw, dashed, rounded corners, color=black, fit={(53a)(53b)(53c)(53d)}
] (expmem) {};
\node [anchor=south, at=(expmem.north)] (expmemlab) {ExplicitMemory};
%% Shadow node -- instead of fitting with 'inner ysep'
\node [anchor=south, at=(kernels.south)] (shadow) {\phantom{KESOACS}};
%%
% Passes
\node[draw, dashed, rounded corners, color=black, inner xsep=3mm, %inner ysep=3mm,
  fit={(51a)(51b)(51c)(52a)(52b)(52c)(52d)(53a)(53b)(53c)(53d)(soacslab)(kernelslab)(expmemlab)(shadow)}
] (passes) {};
\node [anchor=south, at=(passes.north)] (passeslab) {Transformations + Optimisations};
\end{tikzpicture}
%

\end{center}
\caption{Overview of transformations and
  optimisations. Note, that this is only a selection of the
  actual passes performed by the compiler, and that there
  may be passes in between the displayed
  passes. Furthermore, after each change of representation
  the program is type checked.}
\label{fig:passes}
\end{figure}
%


\subsection{Source Language}

%
\begin{figure}
\begin{equation}
\begin{array}{lrlr}
x & ::= & id & \text{(Variable names)} \\
f & ::= & id & \text{(Function names)}
\end{array}
\end{equation}
\caption{Grammar for the Futhark source language.}
\label{fig:futhark_src}
\end{figure}
%


\subsection{Core Language}

%
\begin{figure}
\begin{equation}
\begin{array}{lrlr}
\multicolumn{4}{c}{\text{(Basic values)}} \\
f & ::= & \textbf{id} & \text{(Function names)} \\
x & ::= & \textbf{id} & \text{(Variable names)} \\
c & ::= & \textbf{const} & \text{(Constant values)} \\
\\
\multicolumn{4}{c}{\text{(Types)}} \\
t & ::= & \mathtt{i32}~|~\mathtt{f32}~|~\mathtt{bool}~|~\dots & \text{(Built-in types)} \\
\tau & ::=       & \makebox[10pt][c]{$t$}~|~\texttt{[]}\tau & \text{(Scalar or array type)} \\
\hat{\tau} & ::= & \makebox[10pt][c]{$\tau$}~|~^*\tau & \text{(Nonunique or unique type)} \\
\rho & ::= & (\tau_1, \dots, \tau_n) & \text{(Tuple type)} \\
\hat{\rho} & ::= & (\hat{\tau_1}, \dots, \hat{\tau_n}) & \text{(Nonunique or unique tuple type)} \\
%\phi & ::= & \rho_1 \rightarrow \dots \rightarrow \rho_n & \text{(Function type -- not used)} \\
%\multicolumn{4}{l}{\text{(Also, it couldn't the function use existential types?)}} \\
\\
\multicolumn{4}{c}{\text{(Patterns)}} \\
p & ::= & (x_1 : \tau_1)~\dots~(x_n : \tau_n) & \text{(Let or lambda pattern)} \\
\hat{p} & ::= & (x_1 : \hat{\tau}_1)~\dots~(x_n : \hat{\tau}_n) & \text{(Function pattern)} \\
\\
\multicolumn{4}{c}{\text{(Expressions)}} \\
%\multicolumn{4}{l}{\text{(BasicOp)}} \\
e & ::= & c & \text{(Constant)} \\
& | & x & \text{(Variable)} \\
& | & (x_1, \dots, x_n) & \text{($n$-tuple)} \\
& | & e_1 \oplus e_2 & \text{(Scalar (??) binary operator)} \\
& | & x[x_1, \dots, x_n] & \text{(Array indexing)} \\
& | & x~\mathtt{with}~[e_1, \dots, e_n]~\leftarrow~e & \text{(In-place update)} \\
& | & \mathtt{let}~p~=~e_1~\mathtt{in}~e_2 & \text{(Let binding)} \\
& | & \mathtt{iota}~e & \text{(Iota)} \\
& | & \mathtt{replicate}~e_1~e_2 & \text{(Replicate)} \\
%\multicolumn{4}{l}{\text{(ExpT)}} \\
& | & f~e_1~\dots~e_n & \text{(Function call -- Apply)} \\
& | & \mathtt{if}~e_1~\mathtt{then}~e_2~\mathtt{else}~e_3  & \text{(Branch -- If)} \\
& | & \mathtt{loop}~\hat{p}~=~e_1~\mathtt{for}~x_1<x_2~\mathtt{do}~e_2 & \text{(Loop -- DoLoop)} \\
%\multicolumn{4}{l}{\text{(Op)}} \\
& | & op~e_1~\dots~e_n & \text{(Operator call)} \\
\\
\multicolumn{4}{c}{\text{(Program and functions)}} \\
prog & ::= & fundec~~prog~|~\epsilon & \text{(Program)} \\
fun & ::= & \mathtt{let}~f~\hat{p}~:~\hat{\tau}~=~e& \text{(Named function)}
\end{array}
\end{equation}
\caption{Selected parts of the grammar for the core Futhark
  intermediate representation (IR). Note, that $op$ may
  change with the representation.}
\label{fig:futhark_core}
\end{figure}
%

\subsection{In-place updates}
\subsection{Uniqueness types}
