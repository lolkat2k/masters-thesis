%
\begin{figure}
\begin{center}
\begin{tikzpicture}[
      every node/.style={minimum size=1cm-\pgflinewidth, outer sep=0pt, font=\scriptsize}]
% Top - Host
\filldraw[fill=light-gray, draw=black] (0,2) rectangle (6,3);
\filldraw[fill=dark-gray, draw=black] (0,2) rectangle (0.5,3);
\filldraw[fill=dark-gray, draw=black] (1.5,2) rectangle (3,3);
\filldraw[fill=dark-gray, draw=black] (5,2) rectangle (6,3);
\node[fit={(0,2)(6,3)}] (upper) {};
\node[anchor=east, at=(upper.west)] (hostlab) {Host};
% Bottom - Device
\filldraw[fill=light-gray, draw=black] (0,0) rectangle (6,1);
\filldraw[fill=dark-gray, draw=black] (0.5,0) rectangle (1.5,1);
\filldraw[fill=dark-gray, draw=black] (3,0) rectangle (5,1);
\node[fit={(0,0)(6,1)}] (lower) {};
\node [anchor=east, at=(lower.west)] (devlab) {Device};
% Arrows
\draw[-latex] (0.5,2) -- node[above, rotate=90, align=center] {Data \\ + Code} (0.5,1);
\draw[-latex] (1.5,1) -- node[above, rotate=90] {Data} (1.5,2);
%
\draw[-latex] (3,2) -- node[above, rotate=90, align=center] {Data \\ + Code} (3,1);
\draw[-latex] (5,1) -- node[above, rotate=90] {Data} (5,2);
% Time line
\draw[-latex] ([yshift=0.25cm]upper.north west) -- node[above] {Time} ([yshift=0.25cm]upper.north east);
\draw[latex-latex] ([xshift=0.25cm]upper.north east) -- node[below, rotate=90] {Data} ([xshift=0.25cm]lower.south east);
\end{tikzpicture}
\end{center}
\caption{Heterogeneous programming. The host is transferring
  code and data to the device, and the device transfers back
  data to the host. Here the host is launching two
  kernels. The areas colored with a darker color is
  indicating where the computation is. The host is always in
  control of the device.}
\label{fig:heterogeneous}
\end{figure}
%
