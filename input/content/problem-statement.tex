\chapter{The Problem}
\label{chap:theproblem}

In this section we will describe the problem and give a
brief outline of the research. \todo{For now this is just my thesis description}


\section{Motivation}

\todo{functional languages?} In data-parallel computing a
scatter operation receives as arguments an original array,
an array of indices, and an array of values, and it updates
the elements of the original array at the corresponding
indices with the new provided values. In pseudo-code this
can be written as:
%
\begin{lstlisting}
for(int i=0; i<N; i++) {
  ind, val = f(xs[i])
  ys[i] = val
}
\end{lstlisting}
%
where \texttt{xs} is an input array, \texttt{f} is a bucket
function producing an index and a value, and \texttt{ys} is
the result array.

One can generalize the scatter construct to
represent a fusion between a map and a scatter, in which the
mapped function transforms some arbitrary type \texttt{a}
into index-value pairs of type \texttt{(int, b)}, and the
scatter updates the original array based on these
index-value pairs. The advantage is that the fused
map-scatter does not need to materialize the index-value
arrays in memory.


\section{Problem Statement}

Currently, the intermediate representation of scatter in the
Futhark compiler has the following type signature:
%
\begin{equation}
  \mathtt{scatter^{IR}_{U}} \ : \
  \mathtt{[m]b} \rightarrow
  \mathtt{(a \rightarrow (int, b))} \rightarrow
  \mathtt{[n]a} \rightarrow
  \mathtt{[m]b}
\end{equation}
%
where $\mathtt{[m]b}$ is understood as an array of type
$\mathtt{b}$ and size $m$, and $n \leq m$. An example of its
semantics, assuming a tuple of array representation and type
$\mathtt{a = (int, b)}$, is given by:
%
\begin{equation}
  \mathtt{scatter^{IR}_{U}} \
  \mathtt{[b_0, \dots, b_{m-1}]} \
  \mathtt{id} \
  \mathtt{[0, 2, 4, 6]} \
  \mathtt{[a_0, a_2 , a_4 ,a_6]}
\end{equation}
%
which results in
%
\begin{equation}
  \mathtt{[a_0, b_1, a_2, b_3, a_4, b_5, a_6, b_7, b_8,
      \dots, b_{m-1}]}
\end{equation}
%

One problem however, is that the current
$\mathtt{scatter^{IR}_{U}}$ has a nondeterministic semantics
when there are several to-be-updated values corresponding to
the same index, i.e., the when the index array contains
duplicates. For example, if the map results in $\mathtt{[(0,
    3.0), (0, 4.0), \dots]}$ then the first element of the
result array may be either $\mathtt{3.0}$ or $\mathtt{4.0}$
depending on which update gets executed first. The second
problem is that the current $\mathtt{scatter^{IR}_{U}}$
lacks the ability to combine several values corresponding to
the same index as it will simply overwrite that element.

This thesis studies the feasibility of generalizing the
scatter construct with such support, which for example would
allow efficient computation of histograms. The starting
point is to generalize $\mathtt{scatter^{IR}_{U}}$ as below:
%
\begin{equation}
  \mathtt{scatter^{IR}_{H}} \ : \
  \mathtt{[m]b} \rightarrow
  \mathtt{(b \rightarrow b \rightarrow b)} \rightarrow
  \mathtt{b} \rightarrow
  \mathtt{(a \rightarrow (i32, b))} \rightarrow
  \mathtt{[n]a} \rightarrow
  \mathtt{[m]b}
\end{equation}
%
such that it, in addition, takes as input an associative and
commutative operator and identity element, i.e., a
commutative monoid. The new scatter would have the following
data-parallel semantics:
%
\begin{enumerate}
  \item The map is computed and results, semantically, in an
    array of index-value pairs.
  \item The latter is sorted with respect to indices, and
  \item Re-organized as a two-dimensional irregular array in
    which the segments (rows) correspond to values that
    share the same index.
  \item An irregular segmented reduction, with a specified
    binary associative operator, would compute the
    to-be-updated value for each index, and ultimately
  \item Each such index of the original array is updated to
    the previously computed combined-value.
\end{enumerate}
%

Note however that while implementation as above is possible,
it is also very inefficient because it requires several
(segmented) scans which are slow on GPU hardware. We aim
instead at a code generation that is based on atomic
updates, and which, we believe will provide significant
performance gains.

Finally, the observant reader may notice that the new
scatter is not entirely a generalization of the old scatter:
There does not seem to exist an associative and commutative
binary operator that can match the update semantics of
$\mathtt{scatter^{IR}_{U}}$.


\section{Brief outline of research}

Dunno what this should contain yet. Maybe just some forward
references like in the intro but more detailed and using the
language/knowledge we have learned in the Background
chapter.
