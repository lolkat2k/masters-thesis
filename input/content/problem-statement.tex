\chapter{Problem Statement}

What we are doing look like this in imperative code:
%
\begin{verbatim}
for(int i=0; i<N; i++) {
  idx, val = f(image[i]);
  hist[idx] = hist[idx] 'op' val
}
\end{verbatim}
%
where \texttt{image} is an array of size $N$, and
\texttt{'op'} is some associative and commutative
operator.

If we compute this sequentially memory accesses and ordering
of the operands are deterministic on duplicate indices, and
thus no problems arise. But if we compute it in parallel, we
encounter two main problems:
%
\begin{description}
  \item[Race conditions:] More than one data point may
    correspond to the same bin which obviously produces race
    conditions. We try to solve this by using atomic
    operations.

  \item[Reordering of operands:] Due to the GPUs scheduling
    of threads permutation of operations may occur. We solve
    this by requering that the combining operator is both
    associative and commutative.
\end{description}
%
