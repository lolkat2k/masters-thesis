\chapter{Introduction}

%Something general about parallel computing. Maybe something
%about meta-computation, Robert Glück style.

This thesis describes the development and implementation of
a new language construct in the programming language Futhark
and its optimising compiler. The construct provides a way to
efficiently compute what is referred to as \emph{general
  reductions}. Before we dive into details let us gain some
intuition about the idea and explore on a very high level
some of the problems we need to address. \todo{Need for speed and GPUs are good? Justification for the construct. Maybe after the example?}

Say you have an image consisting of individual pixels each
having a color, and you want to count the occurence of each
color in the image. For the sake of simplicity assume that
we have only four colors: blue, green, red, and yellow. The
simplest algorithm would be to process one pixel at a time:
determine its color and increment the corresponding counter
by one until all pixels in the image have been
processed. The result of counting can be visualised as a
histogram as shown in Figure~\ref{fig:problem}.
%
\begin{figure}
\begin{center}
\begin{tabular}{cc}
\begin{minipage}[c]{0.35\linewidth}
\begin{tikzpicture}[remember picture,
      every node/.style={minimum size=1cm-\pgflinewidth, outer sep=0pt, font=\scriptsize}]
% Bottom row
\node[fill=green!40, anchor=south west] at (0,0) {0};
\node[fill=green!40, anchor=south west] at (1,0) {1};
\node[fill=yellow!40, anchor=south west] at (2,0) {2};
\node[fill=red!40, anchor=south west] at (3,0) {3};
% Lower middle row
\node[fill=yellow!40, anchor=south west] at (0,1) {4};
\node[fill=yellow!40, anchor=south west] at (1,1) {5};
\node[fill=green!40, anchor=south west] at (2,1) {6};
\node[fill=yellow!40, anchor=south west] at (3,1) {7};
% Upper middle row
\node[fill=blue!40, anchor=south west] at (0,2) {8};
\node[fill=blue!40, anchor=south west] at (1,2) {9};
\node[fill=green!40, anchor=south west] at (2,2) {10};
\node[fill=yellow!40, anchor=south west] at (3,2) {11};
% Top row
\node[fill=green!40, anchor=south west] at (0,3) {12};
\node[fill=yellow!40, anchor=south west] at (1,3) {13};
\node[fill=red!40, anchor=south west] at (2,3) {14};
\node[fill=red!40, anchor=south west] at (3,3) {15};
%
\draw[step=1cm,color=black] (0,0) grid (4,4);
%
\node[fit={(0,0)(4,4)}, xshift=0.25cm] (P1L) {};
\end{tikzpicture}
\end{minipage}

 & \begin{minipage}{0.35\linewidth}
\begin{tikzpicture}[remember picture,
      every node/.style={minimum size=1cm, outer sep=0pt, font=\scriptsize}]
% Grid
\draw[-, step=1cm, color=black] (0,0) grid (4,0);
\draw[->, step=1cm, color=black] (0,0) grid (0,3.25);
% Boxes
\newcommand{\mysep}{3pt}
\draw[fill=blue!40]   (0cm + \mysep, 0cm) rectangle node{} +(1cm - 2*\mysep, 1cm);
\draw[fill=green!40]  (1cm + \mysep, 0cm) rectangle node{} +(1cm - 2*\mysep, 2.5cm);
\draw[fill=red!40]    (2cm + \mysep, 0cm) rectangle node{} +(1cm - 2*\mysep, 1.5cm);
\draw[fill=yellow!40] (3cm + \mysep, 0cm) rectangle node{} +(1cm - 2*\mysep, 3cm);
% X-Labels
\draw (0, 0.5)-- +(-0.2, 0) node[left, xshift=0.25cm]{$1$};
\draw (0, 1)-- +(-0.2, 0) node[left, xshift=0.25cm]{$2$};
\draw (0, 1.5)-- +(-0.2, 0) node[left, xshift=0.25cm]{$3$};
\draw (0, 2)-- +(-0.2, 0) node[left, xshift=0.25cm]{$4$};
\draw (0, 2.5)-- +(-0.2, 0) node[left, xshift=0.25cm]{$5$};
\draw (0, 3)-- +(-0.2, 0) node[left, xshift=0.25cm]{$6$};
%\draw (0, 3.5)-- +(-0.2, 0) node[left]{$7$};
%\draw[thin,gray] (0,0.5)-- +(4,0);
\draw[thin,gray] (0,1)-- +(4,0);
%\draw[thin,gray] (0,1.5)-- +(4,0);
\draw[thin,gray] (0,2)-- +(4,0);
%\draw[thin,gray] (0,2.5)-- +(4,0);
\draw[thin,gray] (0,3)-- +(4,0);
%\draw[thin,gray] (0,3.5)-- +(4,0);
% X-Labels
\draw (0.5,0) -- +(0,-0.2) node[below, yshift=0.25cm]{Blue};
\draw (1.5,0) -- +(0,-0.2) node[below, yshift=0.25cm]{Green};
\draw (2.5,0) -- +(0,-0.2) node[below, yshift=0.25cm]{Red};
\draw (3.5,0) -- +(0,-0.2) node[below, yshift=0.25cm]{Yellow};
\path (current bounding box.north west) -- (current bounding box.south west) coordinate[midway] (P1R); % this should be a fit
\end{tikzpicture}
\end{minipage}

\end{tabular}
\tikz[overlay, remember picture]{
  \draw[-latex, thick] (P1L.east) |- (P1R.west) {};
}
\caption{A histogram is simply a visualisation of the
  distribution of data.}
\label{fig:problem}
\end{center}
\end{figure}
%

The described algorithm is here given in imperative C-like
pseudo-code:
%
\begin{lstlisting}
for(int i=0; i<N; i++) {
  col = f(image[i])
  cnt = hist[col]
  hist[col] = cnt + 1
}
\end{lstlisting}
%
where \texttt{image} is an array of size \texttt{N}
containing a flattened version of the image, \texttt{hist}
is an array whose length is equal to the number of unique
colors in the image, and \texttt{f} is a bucket function
computing the index, called a bucket, into \texttt{hist} for
any color in \texttt{image}.

In a sequential (single core?) setting this approach is
completely safe but also very slow. One way to speed up the
process is to realise that determining the color of one
pixel is independent of determining the color of any other
pixel.  Thus, you could half the time taken to compute the
histogram by \emph{cooperating} with a friend by splitting
the image in two equally-sized parts and have your friend
process one half while you process the other half. This idea
is displayed in Figure~\ref{fig:problem2}.

But this approach comes with a cost as it implicitly
introduces \emph{data races}. As an example, assume that you
and your friend processes pixels 9 and 11 at the same time.
Then you will read the current value of the counter for
yellow, 3, increment the value locally by one, 4, and write
back the value simultaneously. Because you and your friend
read the same value the new value after both writes will be
4 but when it should have been 5.
%
\begin{figure}
\begin{center}
\begin{tabular}{ccc}
  \begin{minipage}{0.35\linewidth}
\begin{tikzpicture}[remember picture,
      every node/.style={minimum size=1cm-\pgflinewidth, outer sep=0pt, font=\scriptsize}]
% Top row
\node[fill=blue!40, anchor=south west] at (0,3) {8};
\node[fill=blue!40, anchor=south west] at (1,3) {9};
\node[fill=green!40, anchor=south west] at (2,3) {10};
\node[fill=yellow!40, anchor=south west] at (3,3) {11};
% Bottom row
\node[fill=green!40, anchor=south west] at (0,4) {12};
\node[fill=yellow!40, anchor=south west] at (1,4) {13};
\node[fill=red!40, anchor=south west] at (2,4) {14};
\node[fill=red!40, anchor=south west] at (3,4) {15};
% Grid
\draw[step=1cm,color=black] (0,3) grid (4,5);
\node[fit={(0,3)(4,5)}, xshift=0.25cm] (upper) {};
%% Lower
% Top row
\node[fill=green!40, anchor=south west] at (0,0) {0};
\node[fill=green!40, anchor=south west] at (1,0) {1};
\node[fill=yellow!40, anchor=south west] at (2,0) {2};
\node[fill=red!40, anchor=south west] at (3,0) {3};
% Bottom row
\node[fill=yellow!40, anchor=south west] at (0,1) {4};
\node[fill=yellow!40, anchor=south west] at (1,1) {5};
\node[fill=green!40, anchor=south west] at (2,1) {6};
\node[fill=yellow!40, anchor=south west] at (3,1) {7};
%
\draw[step=1cm,color=black] (0,0) grid (4,2);
\node[fit={(0,0)(4,2)}, xshift=0.25cm] (lower) {};
\end{tikzpicture}
\end{minipage}
 &
  \tikz[remember picture]{\node (mid) {};} &
  \begin{minipage}{0.35\linewidth}
\begin{tikzpicture}[remember picture,
      every node/.style={minimum size=1cm, outer sep=0pt, font=\scriptsize}]
% Grid
\draw[-, step=1cm, color=black] (0,0) grid (4,0);
\draw[->, step=1cm, color=black] (0,0) grid (0,3.25);
% Boxes
\newcommand{\mysep}{3pt}
\draw[fill=blue!40]   (0cm + \mysep, 0cm) rectangle node{} +(1cm - 2*\mysep, 1cm);
\draw[fill=green!40]  (1cm + \mysep, 0cm) rectangle node{} +(1cm - 2*\mysep, 2.5cm);
\draw[fill=red!40]    (2cm + \mysep, 0cm) rectangle node{} +(1cm - 2*\mysep, 1.5cm);
\draw[fill=yellow!40] (3cm + \mysep, 0cm) rectangle node{} +(1cm - 2*\mysep, 3cm);
% Y-Labels
\draw (0, 0.5)-- +(-0.2, 0) node[left, xshift=0.25cm]{$1$};
\draw (0, 1)-- +(-0.2, 0) node[left, xshift=0.25cm]{$2$};
\draw (0, 1.5)-- +(-0.2, 0) node[left, xshift=0.25cm]{$3$};
\draw (0, 2)-- +(-0.2, 0) node[left, xshift=0.25cm]{$4$};
\draw (0, 2.5)-- +(-0.2, 0) node[left, xshift=0.25cm]{$5$};
\draw (0, 3)-- +(-0.2, 0) node[left, xshift=0.25cm]{$6$};
%\draw (0, 3.5)-- +(-0.2, 0) node[left]{$7$};
%\draw[thin,gray] (0,0.5)-- +(4,0);
\draw[thin,gray] (0,1)-- +(4,0);
%\draw[thin,gray] (0,1.5)-- +(4,0);
\draw[thin,gray] (0,2)-- +(4,0);
%\draw[thin,gray] (0,2.5)-- +(4,0);
\draw[thin,gray] (0,3)-- +(4,0);
%\draw[thin,gray] (0,3.5)-- +(4,0);
% X-Labels
\draw (0.5,0) -- +(0,-0.2) node[below, yshift=0.25cm]{Blue};
\draw (1.5,0) -- +(0,-0.2) node[below, yshift=0.25cm]{Green};
\draw (2.5,0) -- +(0,-0.2) node[below, yshift=0.25cm]{Red};
\draw (3.5,0) -- +(0,-0.2) node[below, yshift=0.25cm]{Yellow};
\path (current bounding box.north west) -- (current bounding box.south west) coordinate[midway] (P2R);
\end{tikzpicture}
\end{minipage}

\end{tabular}
\tikz[overlay,remember picture]{
  \draw[-latex, thick] (upper) -| ([yshift=0.5cm]mid.north) |- ([yshift=0.5cm]P2R);
  \draw[-latex, thick] (lower) -| ([yshift=-0.5cm]mid.south) |- ([yshift=-0.5cm]P2R);
}
\caption{Problem of data races. If we split the image in
  half we can process the pixels in half the time, but what
  happens when we simultaneously process two pixels that
  goes into the same bucket?}
\label{fig:problem2}
\end{center}
\end{figure}
%

Such data races can be avoided by using \emph{atomic
  operations}.  This means that one can perform a
read-modify-write operation without being interfered by
someone else trying to read or write the same piece of
data. Effectively, this serialises simultaneous accesses to
the same piece of data. We will go into detail about how
such atomic operations can be used but for now we assume
that they exist and that we can use them.

The observant reader may notice that in the worst case, \todo{and you and your friend uses atomic operations}
where all pixels have the same color, this naive application
of atomic operations corresponds to the sequential version
described in pseudo-code above. That is, the level of
\emph{memory contention} is high.  \todo{Ikke ret godt forklaret / sammenhæng} Hence, we would like to
minimise contention in order to avoid expensive
serialisation of accesses.

One such method for minimising contention is the idea of
\emph{sub-histogramming} which is shown in
Figure~\ref{fig:problem3}. Continuing the example from
before, instead of cooperating on the same histogram you
compute your own local histogram. When all local histograms
are computed one person combines them into the final
histogram. In the worst case describe aboved this approach
will half the time needed to compute the histogram modulo
the time needed to combine the sub-histograms.
%
\begin{figure}
\begin{center}
\begin{tabular}{c}
\begin{tikzpicture}[remember picture,
      every node/.style={minimum size=1cm-\pgflinewidth, outer sep=0pt, font=\scriptsize}]
%%%% IMAGES
%% Left
% Top row
\node[fill=green!40, anchor=south west] at (0,0) {0};
\node[fill=green!40, anchor=south west] at (1,0) {1};
\node[fill=yellow!40, anchor=south west] at (2,0) {2};
\node[fill=red!40, anchor=south west] at (3,0) {3};
% Bottom row
\node[fill=yellow!40, anchor=south west] at (0,1) {4};
\node[fill=yellow!40, anchor=south west] at (1,1) {5};
\node[fill=green!40, anchor=south west] at (2,1) {6};
\node[fill=yellow!40, anchor=south west] at (3,1) {7};
%
\draw[step=1cm,color=black] (0,0) grid (4,2);
\node[fit={(0,0)(4,2)}] (P3IL) {};
%% Right
% Top row
\node[fill=blue!40, anchor=south west] at (5,0) {8};
\node[fill=blue!40, anchor=south west] at (6,0) {9};
\node[fill=green!40, anchor=south west] at (7,0) {10};
\node[fill=yellow!40, anchor=south west] at (8,0) {11};
% Bottom row
\node[fill=green!40, anchor=south west] at (5,1) {12};
\node[fill=yellow!40, anchor=south west] at (6,1) {13};
\node[fill=red!40, anchor=south west] at (7,1) {14};
\node[fill=red!40, anchor=south west] at (8,1) {15};
% Grid
\draw[step=1cm,color=black] (5,0) grid (9,2);
\node[fit={(5,0)(9,2)}] (P3IR) {};
%
%%%% HISTOGRAMS
%% Left
% Grid
\draw[-, step=1cm, color=black] (0,-3) grid (4,-3);
\draw[->, step=1cm, color=black] (0,-3) grid (0,-0.75);
\node[fit={(0,-3)(4,-1)}] (P3HL) {};
% Boxes
\newcommand{\mysep}{3pt}
%\draw[fill=blue!40]   (0cm + \mysep, -3cm) rectangle node{} +(1cm - 2*\mysep, 0cm);
\draw[fill=green!40]  (1cm + \mysep, -3cm) rectangle node{} +(1cm - 2*\mysep, 1.5cm);
\draw[fill=red!40]    (2cm + \mysep, -3cm) rectangle node{} +(1cm - 2*\mysep, 0.5cm);
\draw[fill=yellow!40] (3cm + \mysep, -3cm) rectangle node{} +(1cm - 2*\mysep, 2cm);
% Y-Labels
\draw (0, -2.5)-- +(-0.2, 0) node[left, xshift=0.25cm] {$1$};
\draw (0, -2)-- +(-0.2, 0) node[left, xshift=0.25cm]{$2$};
\draw (0, -1.5)-- +(-0.2, 0) node[left, xshift=0.25cm]{$3$};
\draw (0, -1)-- +(-0.2, 0) node[left, xshift=0.25cm]{$4$};
\draw[thin, gray] (0,-2)-- +(4,0);
\draw[thin, gray] (0,-1)-- +(4,0);
% X-Labels
\draw (0.5,-3) -- +(0,-0.2) node[below, yshift=0.25cm]{Blue};
\draw (1.5,-3) -- +(0,-0.2) node[below, yshift=0.25cm]{Green};
\draw (2.5,-3) -- +(0,-0.2) node[below, yshift=0.25cm]{Red};
\draw (3.5,-3) -- +(0,-0.2) node[below, yshift=0.25cm]{Yellow};
%% Right
% Grid
\draw[-, step=1cm, color=black] (5,-3) grid (9,-3);
\draw[->, step=1cm, color=black] (5,-3) grid (5,-0.75);
\node[fit={(5,-3)(9,-1)}] (P3HR) {};
% Boxes
\draw[fill=blue!40]   (5cm + \mysep, -3cm) rectangle node{} +(1cm - 2*\mysep, 1cm);
\draw[fill=green!40]  (6cm + \mysep, -3cm) rectangle node{} +(1cm - 2*\mysep, 1cm);
\draw[fill=red!40]    (7cm + \mysep, -3cm) rectangle node{} +(1cm - 2*\mysep, 1cm);
\draw[fill=yellow!40] (8cm + \mysep, -3cm) rectangle node{} +(1cm - 2*\mysep, 1cm);
% Y-Labels
\draw (5, -2.5)-- +(-0.2, 0) node[left, xshift=0.25cm]{$1$};
\draw (5, -2)-- +(-0.2, 0) node[left, xshift=0.25cm]{$2$};
\draw (5, -1.5)-- +(-0.2, 0) node[left, xshift=0.25cm]{$3$};
\draw (5, -1)-- +(-0.2, 0) node[left, xshift=0.25cm]{$4$};
\draw[thin, gray] (5,-2)-- +(4,0);
\draw[thin, gray] (5,-1)-- +(4,0);
% X-Labels
\draw (5.5,-3) -- +(0,-0.2) node[below, yshift=0.25cm]{Blue};
\draw (6.5,-3) -- +(0,-0.2) node[below, yshift=0.25cm]{Green};
\draw (7.5,-3) -- +(0,-0.2) node[below, yshift=0.25cm]{Red};
\draw (8.5,-3) -- +(0,-0.2) node[below, yshift=0.25cm]{Yellow};
%%%% ARROWS BETWEEN IMAGES AND HISTOGRAMS
\draw[->, thick] (P3IL) -- (P3HL);
\draw[->, thick] (P3IR) -- (P3HR);
\end{tikzpicture}
 \\
\tikz[remember picture]{\node (p3mid) {};} \\
\begin{tikzpicture}[remember picture,
      every node/.style={minimum size=1cm, outer sep=0pt, font=\scriptsize}]
% Grid
\draw[-, step=1cm, color=black] (0,0) grid (4,0);
\draw[->, step=1cm, color=black] (0,0) grid (0,3.25);
% Boxes
\newcommand{\mysep}{3pt}
\draw[fill=blue!40]   (0cm + \mysep, 0cm) rectangle node{} +(1cm - 2*\mysep, 1cm);
\draw[fill=green!40]  (1cm + \mysep, 0cm) rectangle node{} +(1cm - 2*\mysep, 2.5cm);
\draw[fill=red!40]    (2cm + \mysep, 0cm) rectangle node{} +(1cm - 2*\mysep, 1.5cm);
\draw[fill=yellow!40] (3cm + \mysep, 0cm) rectangle node{} +(1cm - 2*\mysep, 3cm);
% Y-Labels
\draw (0, 0.5)-- +(-0.2, 0) node[left]{$1$};
\draw (0, 1)-- +(-0.2, 0) node[left]{$2$};
\draw (0, 1.5)-- +(-0.2, 0) node[left]{$3$};
\draw (0, 2)-- +(-0.2, 0) node[left]{$4$};
\draw (0, 2.5)-- +(-0.2, 0) node[left]{$5$};
\draw (0, 3)-- +(-0.2, 0) node[left]{$6$};
%\draw (0, 3.5)-- +(-0.2, 0) node[left]{$7$};
\draw[thin,gray] (0,1)-- +(4,0);
\draw[thin,gray] (0,2)-- +(4,0);
\draw[thin,gray] (0,3)-- +(4,0);
\node[fit={(0,0)(4,3)}] (P3fin) {};
\end{tikzpicture}

\end{tabular}
\tikz[overlay,remember picture]{
  \draw[->, thick] (P3HL) |- (p3mid.west) -| (P3fin.north);
  \draw[->, thick] (P3HR) |- (p3mid.east) -| (P3fin.north);
}
\caption{The idea of sub-histogramming. We split the image
  in half and compute two local histograms. When the local
  histograms are done we combine them into the final
  histogram.}
\label{fig:problem3}
\end{center}
\end{figure}
%

Finally, assume that you have much larger image and as many \todo{This was written in a hurry}
people to help you as there are pixels in the image. It is
not hard to imagine that there is a tradeoff between
involving more friends to help you and letting individual
friends process more than one pixel. The idea of individual
friends processing more than one pixel is called
\emph{chunking}. We argue that chunking performs better and
provide the optimal chunk size.


%%% More general: General reduction

The running example is about histograms but the thesis
claims to be about general reductions, so one might think:
How do they relate to each other? A histogram is a specific
case of a general reduction, which structure can be written
as follows:
%% %
%% \begin{figure}
%% \begin{center}
%% \begin{minipage}{0.40\textwidth}
%% \begin{lstlisting}[xleftmargin=0in]
%% for(int i=0; i<N; i++) {
%%   ind, val = f(xs[i])
%%   old = ys[ind]
%%   new = old `op` val
%%   ys[ind] = new
%% }
%% \end{lstlisting}
%% \subcaption{General reduction.}
%% \end{minipage}\hfill
%% \begin{minipage}{0.00\textwidth}
%% \begin{center}
%% %$\Longrightarrow$
%% $\Rightarrow$
%% \end{center}
%% \end{minipage}\hfill
%% \begin{minipage}{0.40\textwidth}
%% \begin{lstlisting}[xleftmargin=0in]
%% for(int i=0; i<N; i++) {
%%   col, _ = f(image[i])
%%   cnt = hist[col]
%%   new = cnt + 1
%%   hist[col] = new
%% }
%% \end{lstlisting}
%% \subcaption{Histogram computation.}
%% \end{minipage}
%% \end{center}
%% \caption{Defining the histogram compution from above in
%%   terms of a general reduction. In the histogram computation
%%   \texttt{val} is always one.}
%% \label{fig:translate}
%% \end{figure}
%% %
%
\begin{lstlisting}
for(int i=0; i<N; i++) {
  ind, val = f(xs[i])
  old = ys[ind]
  new = old `op` val
  ys[ind] = new
}
\end{lstlisting}
%
where \texttt{xs} is the input array of size \texttt{N},
\texttt{ys} is the result array, \texttt{f} is a bucket
function computing an index and a value, and \texttt{op} is
an assocaitive and commutative binary operator. In
particular it uses the operator, \texttt{op}, to combine
\texttt{old} and \texttt{val} into a new value,
\texttt{new}, which is stored at the same index,
\texttt{ind}. If multiple \texttt{x[i]}'s produce the same
index, \texttt{ind}, the corresponding values will be
combined using the original value in \texttt{ys[ind]} as the
base value, hence ``reducing'' the input.
%This is shown in Figure~\ref{fig:translate}.

It is easy to see how the histogram computation can be
described as a general reduction: \texttt{val} will be a
constant one and the operator will be addition.

%% To put our general reduction construct into the landscape we
%% quickly relate it to traditional reductions from the
%% functional programming paradigm. People who know reductions
%% also know general reductions but possibly without being
%% aware: a reduction is simply a general reduction with only
%% one bucket, effectively omitting the index produced by the
%% bucket function, and returning a single value instead of an
%% array of values. \todo{Too vague. Not sure if it should be
%%   included here but it is a nice idea to relate it to other
%%   reductions.}

The main contributions of this thesis are: \todo{Forward references to all chapters?}
%
\begin{itemize}
  \item We explore strategies for implementating
    user-defined atomic operators working on multiple arrays
    in CUDA (Section \ref{sec:critical_sections}). As
    Futhark is ultimately about speed it is desirable to
    investigate the possibility of different code
    generations depending on the input data and the operator
    provided by the user. For example, Futhark internally
    represents arrays-of-tuples as tuples-of-arrays and thus
    calls for different strategies on the GPU.

  \item We explore strategies for sub-histogramming (Section
    \ref{sec:sub_histogramming}) and show that we are able
    to obtain runtimes faster than the naive use of hardware
    optimized instructions (Section
    \ref{sec:performance_experiments}).

  \item In addition we present a heuristic for computing the
    optimal cooperation level in order to minimise memory
    contention (Section \ref{sec:performance_experiments}).
    %\ref{subsec:subhistogramming}).

  \item We argue that chunking performs better than no
    chunking and provide an optimal chunking level. \todo{Is this really what we do?}

  \item We describe the design and implementation of a
    general reduction construct in the programming language
    Futhark and its optimising compiler (Chapter
    \ref{chap:implementation}). The construct is partly
    based on ideas from the existing non-deterministic
    scatter operator, and thus it can be seen as an
    extension that resolves non-determinism (Chapter
    \ref{chap:theproblem}).

  \item (One bullet per optimisation.)

  \item We demonstrate the GPU performance of the code
    generated by the Futhark compiler on a set of benchmarks
    (how many?). Hopefully we show significant
    speedups (Chapter \ref{chap:benchmarks}).
\end{itemize}
%
